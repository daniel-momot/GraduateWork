\chapter*{Заключение} \label{ch-conclusion}
\addcontentsline{toc}{chapter}{Заключение}	% в оглавление 

Заключение (2 -- 5 страниц) обязательно содержит выводы по теме работы, \textit{конкретные
предложения и рекомендации} по исследуемым вопросам. Количество общих выводов
должно вытекать из количества задач, сформулированных во введении выпускной
квалификационной работы.

Предложения и рекомендации должны быть органически увязаны с выводами
и направлены на улучшение функционирования исследуемого объекта. При разработке
предложений и рекомендаций обращается внимание на их обоснованность,
реальность и практическую приемлемость.

Заключение не должно содержать новой информации, положений, выводов и
т. д., которые до этого не рассматривались в выпускной квалификационной работе.
Рекомендуется писать заключение в виде тезисов.

Последним абзацем в заключении можно выразить благодарность всем людям, которые помогали автору в написании ВКР.